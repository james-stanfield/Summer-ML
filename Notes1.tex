\documentclass[14pt, twocolumn]{article}
\usepackage{standalone}
\usepackage{amsmath,amsfonts,amssymb,mathrsfs,commath}
\usepackage{xcolor}{}
\usepackage{latexsym}
\usepackage{mathtools}
\usepackage{preview}
\usepackage[margin = 1in]{geometry}
\usepackage[none]{hyphenat}
\usepackage[]{fancyhdr}


\pagestyle{fancy}
\fancyhead{}
\fancyfoot{}
\fancyhead[L]{James Stanfield}
\fancyhead[C]{Summer Notes}
\fancyhead[R]{43971358}

\parindent 0ex


\newcommand{\kd}[2]{\delta_{#1#2}}
\newcommand{\opsubset}{\overset{\circ}{\subset}}
\newcommand{\R}{\mathbb{R}}
\newcommand{\me}{\mathellipsis}
\newcommand{\intr}{\textrm{Int }}
\newcommand{\pder}[2]{\frac{\partial#1}{\partial#2}}
\newcommand{\cv}[1]{\pder{}{x^#1}}
\newcommand{\id}[1]{\textrm{Id}_{#1}}

\begin{document}


	\begin{center}
		\Large\textbf{Machine Learning: Geometric aspects}\\
		Semester 2 2018
	\end{center}

	\section*{Overview}

	\section*{Smooth Manifolds}

	19/11/2018. Notes from Lee: Intro to smooth manifolds \\

	A topological space $M$ is a topological manifold of dimension $n$ if:\\
	$\bullet$ $M$ is hausdorff\\
	$\bullet$ $M$ is second countable\\
	$\bullet$ $\forall p \in M \ \exists U$ a neighbourhood of $p$, $V \opsubset \R^n$ and a homeomorphism $\psi:U \to V$\\

	$(U , \psi)$ is the coordinate pair, with $U$ the coordinate neighbourhood and $\psi$ the (local) coordinate map. If $\psi(p) = 0$ the chart is centred at $p$. We can construct from a chart that contains $p$, a new chart centred at $p$ by simply subtracting $\psi(p)$. \\

	$\psi(p) = (x^1(p) , \mathellipsis , x^n(p))$ where $(x^1,\mathellipsis,x^n)$ are the local coordinates on $U$.\\

	If $M$ is a topological manifold:\\

	(a) $M$ is locally path connected\\
	(b) $M$ is connected $\iff$ it is bath connected \\
	(c) the components of $M$ are the same as it's path components\\
	(d) $M$ has at most countably many components, each of which is a subset of $M$ and a connected topological manifold \\

	Smooth Manifolds:\\
	We now add a smooth structure that enables calculus.\\
	Note $f: U \subset \R^n \to V \subset \R^m$ is smooth if the partial derivatives of all orders of every component exist.\\

	Let $M$ be a topological manifold. Let $(U,\phi), (V,\psi)$ are be two charts. If $U \cap V \neq \emptyset$, the composite map $\psi \circ \phi^{-1} : \phi(U \cap V) \to \psi(U \cap V)$ is the transition map from $\phi$ to $\psi$. The charts are said to be smoothly compatible if either $U \cap V = \emptyset$ or the transition map $\psi \circ \phi^{-1}$ is a diffeomorphism (Where smoothness is interpreted in the usual sense).\\

	An \textit{atlas} is a collection of charts whose domains cover $M$. An atlas $\mathcal{A}$ is a smooth atlas if every pair of charts is smoothly compatible. Equivalently if for every pair of charts $\psi$ , $\phi$, $\psi \circ \phi^{-1}$ is smooth then the atlas a smooth atlas. $\mathcal{A}$ is maximal if it is not contained in any other smooth atlas.\\

	A \textit{smooth structure} on a topological manifold $M$ is a maximal smooth atlas. A smooth manifold is a pair $(M,\mathcal{A})$. Note it is possible to choose multiple smooth structures not compatible with each other for the same manifold.\\

	Once we choose a chart $(U,\phi)$ on $M$, the coordinate map $\phi : U \to V$ can be thought of as an identification between  $U$ and $V$. We can then represent a point by it's coordinates $(x^1 ,\mathellipsis , x^n) = \phi(p)$ and think of the tuple as being the point itself. $p = (x^1 ,\mathellipsis , x^n)$ in local coordinates. We can also think of the map as being the identity map to simplify notation. e.g. $p = (x,y)$ or $p = (r,\theta)$ \\

	Einstein summation convention:
	We write $\sum_i x^i E_i = x^iE_i$ for example. Basis vectors are writien with lower indicies and components are written with upper indicies.\\

	Smooth manifold examples:\\
	$\bullet$ The zero manifold (countable discrete space).\\
	$\bullet \ \R^n$\\
	$\bullet$ finite dimensional vector spaces\\
	$\bullet \ m \times n$ matricies, $M(m \times n, \R)$ In the case of square matricies simply $M(n,\R)$ \\
	$\bullet$ open sub-manifolds. Let $(M,\mathcal{A})$ be a smooth manifold. 
	\\If $U \opsubset M$ then take $\mathcal{A}_U = \{ (V,\phi) \in \mathcal{A} : V \subset U\}$\\
	$\bullet \ GL(n,\R)$ i.e. matricies with determinant non-zero. Open subset of set of matricies so clearly a smooth manifold.\\
	$\bullet$ Matricies of maximal rank $M_m(m\times n,\R)$\\
	$\bullet \ \mathbb{S}^n$\\
	$\bullet \ \R \mathbb{P}^n$ Real projective plane\\
	$\bullet$ if $M_1 , \mathellipsis , M_k$ are smooth manifolds of dimensions $n_1 , \mathellipsis , n_k$ Then $M_1 \times \mathellipsis \times M_n$ is a smooth manifold with charts of the form $(U_1 \times \mathellipsis \times U_k , \phi_1 \times \mathellipsis \times \phi_k)$\\
	$\bullet$ Grassman Manifolds (important). Let $V$ be an n-dimensional real vector space. If $0 \leq k \leq n$, $G_k(V)$ is the set of all k-dimensional linear subspaces of V.\\

	Need to generalise to manifolds with boundary. We use the closed n-dimensional upper half plane, $\mathbb{H}^n = \{(x^1,\me, x^n) \in \R^n : x^n \geq 0\}$. $\intr\mathbb{H}^n$ and $\partial \mathbb{H}^n$ are the interior and boundaries respectively.\\

	Smooth maps/functions:

	If $M$ is a smooth manifold, $f : M \to \R^k$ is said to be \textit{smooth} if $\forall p \in M \exists (U,\phi)$ for $M$ whose domain contains $p$ such that $f \circ \phi^{-1}$ is smooth on $V = \phi(U) \subset \R^n$. $f \circ \phi^{-1}$ is the \textit{coordinate represention} of f.
	Denote the set of smooth functions $f : M \to \R$ as $C^\infty(M)$. This is a vector space.\\

	Further, if $M , N$ are two smooth manifolds. $F : M \to N$ is a \textit{smooth map} if $\forall p \in M \exists (U,\phi)$ containing $p$ and $(V,\psi)$ containing $F(p)$ such that $F(U) \subset V$ and $\psi \circ F \circ \phi^{-1} : \phi(U) \to \psi(V)$ is smooth. $\hat{F} = \psi \circ F \circ \phi^{-1}$ is the \textit{coordinate representation} of $F$ with respect to the the given coordinates.\\

	Diffeomorphisms:
	$F : M \to N$ is a \textit{diffeomorphism} if it is a smooth, bijective map with a smooth inverse. $M$ is diffeomorphic to $N$ sometimes written $M \approx N$.\\

	$F: M\to N$ is a \textit{local diffeomorphism} if $\forall p \in M \exists$ a neighbourhood $U$ such that $F(U)$ is open in $N$ and $F|_U : U \to F(U)$ is a diffeomorphism. Clearly every local diffeomorphism is an open map. Analogous to topological spaces being identified by homeomorphisms.\\

	Interesting: If $\mathcal{A}_1$ and $\mathcal{A}_2$ are two smooth structures on $\R$, then $\exists$ a diffeomorphism $F:(\R,\mathcal{A}_1) \to (\R,\mathcal{A}_2)$. More generally, every topological manifold of dimension less than or equal to 3 has a smooth structure unique up to diffeomorphism. Higher dimensions the question remains unanswered. For $\R^n$, as long as $n \neq 4$, $\R^n$ has a unique smooth structure up to diffeomorphism. However $\R^4$ has uncountably many smooth structures none of which are diffeomorphic to each other(lol wtf).\\


	Lie groups:\\
	This is a smooth manifold $G$ that is also a group under the multiplication map $m : G \times G \to G$ such that $m(g,h) = gh$ and $i(g) = g^{-1}$ (inversion) are smooth.\\
	Skipped the rest of this chapter, may come back\\

	Tangent Vectors:\\
	If $p$ is a point of $M$, a linear map $X:C^\infty(M) \to \R$ is called a \textit{derivation} at $p$ if:
	$$X(fg) = f(p)Xg + g(p)Xf$$
	$\forall f,g \in C^\infty(M)$. The set of all derivations at $p$ is a vector space called the \textit{tangent space} to $M$ at $p$ denoted by $T_pM$.If $M = \R^n$ this space is isomorphic to the geometric tangent space $\R^n_a = \{(a,v) : v\in \R^n\}$. Can thus visualise tangent vectors as arrows tangent to M whose base points are attatched to M.\\

	For any $a \in \R^n$ the $n$-derivations
	$$\pder{}{x^i}\Big|_a$$
	given by 
	$$\pder{}{x^i}\Big|_a f = \pder{f}{x^i}(a)$$
	form a basis for $T_a(\R^n)$

	If $M , N$ are smooth manifolds and $F:M \to N$ a smooth map, for each $p \in M$ define a map $F_*:T_pM \to T_{F(p)}N$ called the \textit{pushforward} by:
	$$F_*X(f) = X(f \circ F)$$
	if $f \in C^\infty(N)$ then $f\circ F \in C^\infty(M)$\\
	Some properties $F:M \to N, G:N \to P$ are smooth maps:\\
	(a) $F_* : T_pM \to T_{F(P)}N$ is linear.\\
	(b) $(G \circ F)_* = G_* \circ F_* : T_pM \to T_{G\circ F(p)}P$\\
	(c) $(\id{M})_* = \id{T_pM} :T_pM \to T_pM$\\
	(d) If $F$ is a diffeomorphism, then $F_*$ is an isomorphism.\\

	If $p \in M$ and $X \in T_pM$, and $f,g$ are smooth functions on $M$ that agree on some neighborhood of $p$ then $Xf = Xg$.\\

	If $(U,\phi)$ is a smooth coordinate chart on $M$, $\phi$ is a diffeomorphism from $U$ to $V \opsubset \R^n$. $\phi_* : T_pM \to T_{\phi(p)}\R^n$ is an isomorphism. Since this has a basis as above, the pushforward of this basis under $\phi^{-1}_*$ form a basis for $T_pM$. We collapse the notation thusly:

	$$\pder{}{x^i}\Bigg|_p = (\phi^{-1})_* \pder{}{x^i} \Bigg|_{\phi(p)}$$
	This acts on a smooth function $f: U \to \R$ by:
	$$\pder{}{x^i}\Bigg|_p f = \pder{\hat{f}}{x^i}(\hat{p})$$
	Which is the derivative of the coordinate representation of $g$ at the coordinate representation of $p$. Such vectors are called the coordinate vectors at $p$ asssociated with the given coordiante system. In $\R^n$ these correspond exactly to the vetors $e_i$ under the isomorphism $T_a\R^n \leftrightarrow \R_a^n$.\\

	The coordinate vectors form a basis for $T_pM$\\
	$\forall X \in T_pM, X = X^i \cv{i}\big|_p$.\\
	$X^j = (X^i \cv{i})(x^j) = X^j$

	$F_*$ is represented in terms of the coordinate bases by the jacobian matrix of the coordinated representative of $F.$\\

	21/11/2018:\\
	Adding a metric:\\

	\section*{Main paper notes / questions:}

	



	


	\section*{Defintitions/Glossary:}

	%Quotient Space: Let $(X,\tau_X)$ be a topological space and let $\equiv$ be an equivalence relation on X. The quotient space $Y = X / \equiv$ is defined by the set of equivalences classes of elements of $X$:

	%$$Y = \{\{v \in X : v \equiv x\} : x \in X\}$$
	%With the toplogy:
	%$$
	


\end{document}